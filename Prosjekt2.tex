\documentclass[a4paper,12pt, english]{article}
\usepackage[T1]{fontenc}
\usepackage[utf8]{inputenc}
\usepackage{graphicx}
\usepackage{babel}
\usepackage{amsmath}
\usepackage{ulem}
\usepackage{a4wide}
\usepackage{graphicx}
\usepackage{listings}
\usepackage{tabularx}
\usepackage{tabulary}

\title{FYS3150 - Prosjekt 2}
\author{John-Anders Stende}

\begin{document}

\section*{Introduction}

In this project we're going to use Jacobi's eigenvalue algorithm to solve Schr\"odinger's equation for two 
electrons in a three-dimensional harmonic oscillator well with and without a repulsive Coulomb interaction.

To do this, we have to discretize Schr\"odinger's equation and write it as an eigenvalue equation. We assume 
spherical symmetry. For one electron the radial part of Schr\"odinger's equation reads
\[
  -\frac{\hbar^2}{2 m} \left ( \frac{1}{r^2} \frac{d}{dr} r^2
  \frac{d}{dr} - \frac{l (l + 1)}{r^2} \right )R(r) 
     + V(r) R(r) = E R(r).
\]
In our case $V(r)$ is the harmonic oscillator potential $(1/2)kr^2$ with
$k=m\omega^2$ where $\omega$ is the oscillator frequency and $E$ is
the energy of the harmonic oscillator in three dimensions. The quantum number $l$ is the orbital
momentum of the electron. In this project we set $l = 0$.

To write the equation in a more suitable form we substitute 
$R(r) = (1/r) u(r)$ and $\rho = (1/\alpha) r$ where $\alpha$ is a constant with dimension length,
$\rho$ is therefore a dimensionless variable. 
If we also insert $l=0$ and $V(\rho) = (1/2) k \alpha^2\rho^2$ and multiply with 
$2m\alpha^2/\hbar^2$ on both sides we end up with
\[
  -\frac{d^2}{d\rho^2} u(\rho) 
       + \frac{mk}{\hbar^2} \alpha^4\rho^2u(\rho)  = \frac{2m\alpha^2}{\hbar^2}E u(\rho) .
\]
We fix the constant $\alpha$ in the following way:
\[
\frac{mk}{\hbar^2} \alpha^4 = 1,
\]
or 
\[
\alpha = \left(\frac{\hbar^2}{mk}\right)^{1/4}.
\]
Defining 
\[
\lambda = \frac{2m\alpha^2}{\hbar^2}E,
\]
we can rewrite Schr\"odinger's equation as
\[
  -\frac{d^2}{d\rho^2} u(\rho) + \rho^2u(\rho)  = \lambda u(\rho) .
\]
We then discretize the equation with the step length $h$ given by
\[
  h=\frac{\rho_{\mathrm{max}}-\rho_{\mathrm{min}} }{n_{\mathrm{step}}}.
\]
where $\rho_{\mathrm{min}}=0$ and $\rho_{\mathrm{max}}$ needs to be set to a fitting value.
If we insert the standard expression for the second derivate of a function u,
the discretized version of the equation becomes
\[
-\frac{u_{i+1} -2u_i +u_{i-1}}{h^2}+\rho_i^2u_i=-\frac{u_{i+1} -2u_i +u_{i-1} }{h^2}+V_iu_i  = \lambda u_i,
\]
where $V_i=\rho_i^2$ is the harmonic oscillator potential.
To write this as a an eigenvalue problem we define the diagonal matrix element
\[
   d_i=\frac{2}{h^2}+V_i,
\]
and the non-diagonal matrix element 
\[
   e_i=-\frac{1}{h^2}.
\]
which when inserted in the equation becomes
\[
d_iu_i+e_{i-1}u_{i-1}+e_{i+1}u_{i+1}  = \lambda u_i,
\]
Finally we can write the Schr\"odinger equation as an eigenvalue problem:
\begin{equation}
    \left( \begin{array}{ccccccc} d_1 & e_1 & 0   & 0    & \dots  &0     & 0 \\
                                e_1 & d_2 & e_2 & 0    & \dots  &0     &0 \\
                                0   & e_2 & d_3 & e_3  &0       &\dots & 0\\
                                \dots  & \dots & \dots & \dots  &\dots      &\dots & \dots\\
                                0   & \dots & \dots & \dots  &\dots       &d_{n_{\mathrm{step}}-2} & e_{n_{\mathrm{step}}-1}\\
                                0   & \dots & \dots & \dots  &\dots       &e_{n_{\mathrm{step}}-1} & d_{n_{\mathrm{step}}-1}

             \end{array} \right)      \left( \begin{array}{c} u_{1} \\
                                                              u_{2} \\
                                                              \dots\\ \dots\\ \dots\\
                                                              u_{n_{\mathrm{step}}-1}
             \end{array} \right)=\lambda \left( \begin{array}{c} u_{1} \\
                                                              u_{2} \\
                                                              \dots\\ \dots\\ \dots\\
                                                              u_{n_{\mathrm{step}}-1}
             \end{array} \right) 
      \label{eq:sematrix}
\end{equation} 
Boundary conditions: $u = 0$ in both ends, ie. for $i = 0$ and $i = n_{\mathrm{step}}$.

\section*{Algorithm}
Jacobi's algorithm solves this eigenvalue problem by doing a series of similarity transformations
on the matrix above, which I'll call ${\bf A}$. When ${\bf A}$ is real and symmetric as in this
case, linear algebra says that there exists an orthogonal matrix ${\bf S}$ so that
\[
   {\bf S}^T {\bf A} {\bf S} = diag(\lambda_1, ... ,\lambda_n)
\]
and we end up with ${\bf A}$ containing its eigenvalues on the diagonal.

The algorithm's task is therefore to transform A into a (nearly) diagonal matrix. 
That can be obtained by reducing the Frobenius norm of the off-diagonal elements of A
with every transformation until the norm is less than a given tolerance $\epsilon$. 
The transformation matrix ${\bf S}$ used to do this is called a Givens rotation matrix, which
have the following properties
\[
s_{kk} = s_{ll} = \cos\theta,\ s_{kl} = -s_{lk} = -\sin\theta,\ s_{ii} = -s_{ii} = 1\ for\  i \neq k\ and\ i \neq l
\]
A rotation ${\bf B} = {\bf S}^T {\bf A} {\bf S}$ transforms the off-diagonal elements
in the following way (and also ensures that $||{\bf B}||_F^2 < ||{\bf A}||_F^2$)
\[
b_{kl} = (a_{kk}-a_{ll}) \cos\theta \sin\theta + a_{kl}(\cos^2\theta-\sin^2\theta)
\]
The recipe is now to choose $\theta$ so that all $b_{kl}$ become zero. If we set this equation equal to zero 
and define 
$\tan\theta = t= s/c$, with $s=\sin\theta$ and $c=\cos\theta$ we end up with
(using $\cos^2\theta - \sin^2\theta = \cos2\theta$ and $\cos\theta\sin\theta = (1/2)\sin2\theta$)
\[
\cot 2\theta=\tau = \frac{a_{ll}-a_{kk}}{2a_{kl}}.
\]
and then using $\cot 2\theta=1/2(\cot \theta-\tan\theta)$ we obtain the quadratic equation
\[
t^2+2\tau t-1= 0,
\]
resulting in 
\[
  t = -\tau \pm \sqrt{1+\tau^2},
\]
and $c$ and $s$ are easily obtained via
\[
   c = \frac{1}{\sqrt{1+t^2}},
\]
and $s=tc$. We should choose t to be the smaller of the roots since this maximizes c, which in turn minimizes
the difference between the Frobenius norm of ${\bf B}$ and ${\bf A}$
\[
||{\bf B}-{\bf A}||_F^2=4(1-c)\sum_{i=1,i\ne k,l}^n(a_{ik}^2+a_{il}^2) +\frac{2a_{kl}^2}{c^2}.
\]
The algorithm can be implemented as a while-test where we compare the norm of the newly computed off-diagonal
matrix elements with $\epsilon$. This can however be replaced by the simpler test $max(a^2_{ij} > \epsilon$.
For each iteration we therefore find the off-diagonal element with largest absolute value and use this
to compute $t$, $c$ and $s$ so that we can rotate matrix ${\bf A}$ until the test above is fulfilled. 

\section*{Results for one electron}
To verify the results we compare them with the analytic solution for the energies of one electron in a 3d 
harmonic
oscillator potential. The energies are
\[
E_{nl}=  \hbar \omega \left(2n+l+\frac{3}{2}\right),
\]
with $n=0,1,2,\dots$ and $l=0,1,2,\dots$.
For $l = 0$ and according to our definition of $\lambda$ the eigenvalues are 
\[
\lambda_0=3,\ \lambda_1=7,\ \lambda_2=11,\ \dots 
\]
When I set $\rho_{\mathrm{max}} = 4.5$ and compile my script c++ main.cpp -larmadillo -llapack -lblas
I get the following results for the three lowest eigenvalues (printed to terminal):

n = 50; Time elapsed Arma: 0; Eigenvalues Arma: 2.99757, 6.98794, 10.9759; Time elapsed Jacobi: 0;
Eigenvalues Jacobi: 2.99757, 6.98794, 10.9759; Number of iterations: 4070

n = 100; Time elapsed Arma: 0; Eigenvalues Arma: 2.99938, 6.99702, 10.9981; Time elapsed Jacobi: 2;
Eigenvalues Jacobi: 2.99938, 6.99702, 10.9981; Number of iterations: 16563

n = 150; Time elapsed Arma: 0; Eigenvalues Arma: 2.99972, 6.99874, 11.0023; Time elapsed Jacobi: 12;
Eigenvalues Jacobi: 2.99972, 6.99874, 11.0023; Number of iterations: 37382

n = 200; Time elapsed Arma: 0; Eigenvalues Arma: 2.99984, 6.99934, 11.0038; Time elapsed Jacobi: 38;
Eigenvalues Jacobi: 2.99984, 6.99934, 11.0038; Number of iterations: 66935

I've compared the eigenvalues obtained by Jacobi's algorithm to the ones obtained by the Armadillo 
function eig_sym for different $n$ (dimension of matrix ${\bf A}$). We see that both algorithms creates the 
exact
same eigenvalues for each $n$, which suggests that my algorithm works well. For $n = 100$ and $n = 150$ 
the eigenvalues 
have four leading digits except $\lambda_1$, which has three. For $n = 200$ however, all the 
eigenvalues have four
correct digits. 

The Armadillo function is howerever much faster than mine, as the time elapsed is less than a 
second for all 
the $n's$ I 
have tested. I've also listed the number of iterations used by Jacobi's algorithm, which we 
see approximately quadruples
when $n$ doubles.

\section*{Two electrons with repulsive Coulomb interaction}
To study this problem we need to rewrite the Schr\"odinger equation. We introduce the relative 
coordinate ${\bf r} = {\bf r}_1-{\bf r}_2$
and the center-of-mass coordinate ${\bf R} = 1/2({\bf r}_1+{\bf r}_2)$.
With these new coordinates, the radial Schr\"odinger equation reads
\[
\left(  -\frac{\hbar^2}{m} \frac{d^2}{dr^2} -\frac{\hbar^2}{4 m} \frac{d^2}{dR^2}+ \frac{1}{4} k r^2+  kR^2\right)u(r,R)  = E^{(2)} u(r,R).
\]
where the energies and masses of the electrons are summarized and where we're dealing with a two-electron 
wave function
$u(r,R)$ and two-electron energy $E^{(2)}$. We then add the repulsive Coulomb interaction 
\[
V(r_1,r_2) = \frac{\beta e^2}{|{\bf r}_1-{\bf r}_2|}=\frac{\beta e^2}{r},
\]
and make the same substitution $\rho = r/\alpha$ as before, arriving
\[
  -\frac{d^2}{d\rho^2} \psi(\rho) 
       + \frac{1}{4}\frac{mk}{\hbar^2} \alpha^4\rho^2\psi(\rho)+\frac{m\alpha \beta e^2}{\rho\hbar^2}\psi(\rho)  = 
\frac{m\alpha^2}{\hbar^2}E_r \psi(\rho) .
\]
We define a 'frequency' reflecting the strength of the oscillator potential
\[
\omega_r^2=\frac{1}{4}\frac{mk}{\hbar^2} \alpha^4,
\]
and fix the constant $\alpha$ in a similar way, defining
\[
\lambda = \frac{m\alpha^2}{\hbar^2}E,
\]
and Schr\"odinger's equation now reads 
\[
  -\frac{d^2}{d\rho^2} \psi(\rho) + \omega_r^2\rho^2\psi(\rho) +\frac{1}{\rho}\psi(\rho) = \lambda \psi(\rho).
\]
To solve this equation I use the same code as before while changing the potential to $\omega_r^2\rho^2+1/\rho$.

My results for $n = 150$ and $\rho_{max} = 4.5$ are as follows (compiled in the same way):

omega_r = 0.01; Eigenvalues Jacobi: 0.98635, 2.61287, 5.14763

omega_r = 0.5; Eigenvalues Jacobi: 2.23596, 4.26942, 6.86334

omega_r = 1; Eigenvalues Jacobi: 4.05759, 7.90858, 11.8246

omega_r = 5; Eigenvalues Jacobi: 17.4414, 37.0376, 56.7704

We see that the eigenvalue for the ground state increases with increasing $\omega_r$. The two-electron energy is 
in other words larger when the oscillator potential is stronger, as expected. When $\omega_r$ increases 
the potential curve becomes steeper, and the electrons will have a larger energy compared to a more 
shallow potential. 

The above equation has answers in closed form only for specific oscillator frequencies, 
but I've not been able to access the article linked in the project text, which means that I 
don't have any energy values to compare with. 

\subsection*{Plots of wavefunctions}
The wavefunctions $u^{(\lambda_j)}(\rho)$ are the eigenvectors of ${\bf A}$, one for each eigenvalue. 
According to linear algebra the columns of ${\bf S}$ will contain the eigenvectors of ${\bf A}$ when 
all the necessary rotations are performed:
\[   
   {\bf S}^T {\bf A} {\bf S} = {\bf D} \rightarrow {\bf A} {\bf S} = {\bf S} {\bf D}
\]
which is equivalent to
\[
   {\bf A}{\bf S(:,j)} = \lambda_j{\bf S(:,j)} \ for j = 1:n
\]
The rotation of ${\bf S}$ can be thought of as ${\bf I}{\bf S} = {\bf S}$, and the resultant transformation can be implemented in the script, although I got wrong results when I tried this. Instead I've used the Armadillo function 
eig_sym to also obtain the eigenvectors. I've written the eigenvectors/wavefunctions corresponding to 
the three lowest eigenvectors to a file, and used Python to plot them. 

\begin{figure}[hbt]
\begin{center}
\fbox{\includegraphics[width=\textwidth]{fig1.eps}}
\end{center}
\end{figure} 

\begin{figure}[hbt]
\begin{center}
\fbox{\includegraphics[width=\textwidth]{fig2.eps}}
\end{center}
\end{figure} 

\begin{figure}[hbt]
\begin{center}
\fbox{\includegraphics[width=\textwidth]{fig3.eps}}
\end{center}
\end{figure} 

\begin{figure}[hbt]
\begin{center}
\fbox{\includegraphics[width=\textwidth]{fig4.eps}}
\end{center}
\end{figure} 

\begin{figure}[hbt]
\begin{center}
\fbox{\includegraphics[width=\textwidth]{fig5.eps}}
\end{center}
\end{figure} 

For both one and two electrons we see that the wavefunctions have one more node for each excited state, 
which is in fact predicted by quantum mechanics. For two electrons, the wavefunction depends on the 
strength $\omega_r$ of the harmonic oscillator potential. For $\omega_r = 0.01$ the wavefunction 
stretches further out from zero, which is natural considering that the "equilibrium force" in this 
case is weaker. The wavefunction is generally wider when we have repulsive Coulomb interaction 
(except for $\omega_r = 5.0$, when the potential is so strong that it keeps the electrons close to zero) 
which is obviously due to electrons with the same charge repulsing each other. 

\section*{List of programs}
The c++ script that solves Schr\"odinger's equation is \textit{main.cpp}, while the python 
script \textit{plot2.py} plots the wavefunctions. These programs can be found here: 
https://github.com/johnands/Project2. 






\end{document}




